\documentclass[a4paper,12pt]{article}
\usepackage[utf8]{inputenc}
\usepackage[T1]{fontenc}
\usepackage[polish]{babel}
\usepackage{graphicx}
\usepackage{geometry}
\usepackage{hyperref}
\usepackage{float}
\usepackage{titlesec}
\usepackage{xcolor}

% Konfiguracja marginesów
\geometry{
 a4paper,
 total={170mm,257mm},
 left=20mm,
 top=20mm,
}

% Konfiguracja linków
\hypersetup{
    colorlinks=true,
    linkcolor=black,
    filecolor=magenta,      
    urlcolor=blue,
}

\title{\textbf{Instrukcja konfiguracji połączenia zdalnego (VPN + WinSCP)}}
\author{Maria Junak}
\date{\today}

\begin{document}

\maketitle

\begin{abstract}
Niniejszy dokument przedstawia procedurę konfiguracji bezpiecznego połączenia z siecią uniwersytecką (Instytut Matematyczny). Instrukcja obejmuje zestawienie tunelu VPN oraz obsługę programu WinSCP w celu przesyłania plików między komputerem lokalnym a serwerem zdalnym.
\end{abstract}

\tableofcontents
\newpage

\section{Konfiguracja i uruchomienie VPN}

Aby uzyskać dostęp do wewnętrznej sieci laboratoriów (adresacja 192.168.x.x), konieczne jest zestawienie tunelu VPN. Bez tego kroku serwery plików są niewidoczne z internetu.

\subsection{Dodawanie połączenia w systemie Windows}
W ustawieniach sieci VPN należy dodać nową konfigurację dla serwera Instytutu.
\begin{itemize}
    \item \textbf{Dostawca:} Windows (wbudowane)
    \item \textbf{Adres serwera:} \texttt{156.17.86.7}
    \item \textbf{Typ:} Automatyczny
\end{itemize}

\begin{figure}[H]
    \centering
    \includegraphics[width=0.75\textwidth]{screenshots/vpn_setup.png}
    \caption{Konfiguracja parametrów połączenia VPN.}
\end{figure}

\subsection{Nawiązywanie połączenia}
Po zatwierdzeniu ustawień, należy połączyć się z siecią. Status "Połączono" świadczy o poprawnym zestawieniu tunelu.

\begin{figure}[H]
    \centering
    \includegraphics[width=0.5\textwidth]{screenshots/vpn_connected.png}
    \caption{Prawidłowo zestawione, aktywne połączenie VPN.}
\end{figure}

\newpage

\section{Transfer plików za pomocą WinSCP}

Po uzyskaniu dostępu do sieci wewnętrznej, skonfigurowano klienta SFTP (WinSCP) do zarządzania plikami.

\subsection{Logowanie do serwera}
W oknie logowania wprowadzono dane dostępowe do konkretnego stanowiska laboratoryjnego:
\begin{itemize}
    \item \textbf{Nazwa hosta:} \texttt{192.168.50.144} (lub inny z przydzielonej puli)
    \item \textbf{Użytkownik i hasło:} Dane konta w pracowni.
\end{itemize}

\begin{figure}[H]
    \centering
    \includegraphics[width=0.75\textwidth]{screenshots/winscp_login.png}
    \caption{Okno logowania programu WinSCP z wybranym protokołem SFTP.}
\end{figure}

\subsection{Przesyłanie plików}
Logowanie przebiegło pomyślnie. Interfejs programu został podzielony na dwie sekcje:
\begin{itemize}
    \item \textbf{Lewy panel:} Pliki lokalne (Twój komputer).
    \item \textbf{Prawy panel:} Pliki zdalne (Serwer Instytutu).
\end{itemize}

Transfer plików odbywa się poprzez przeciągnięcie ikony pliku z jednego panelu do drugiego (metoda \textit{Drag \& Drop}). Poniższy zrzut ekranu prezentuje zalogowanego użytkownika oraz operację przesyłania pliku testowego.

\begin{figure}[H]
    \centering
    \includegraphics[width=0.9\textwidth]{screenshots/winscp_main.png}
    \caption{Główne okno WinSCP. Widoczna aktywna sesja użytkownika oraz proces transferu pliku testowego.}
\end{figure}

\section{Podsumowanie}
Proces konfiguracji zakończył się sukcesem. Połączenie VPN jest stabilne, a klient WinSCP umożliwia pełny dostęp do zasobów plikowych na koncie studenckim.

\end{document}