\documentclass[a4paper,12pt]{article}
\usepackage[utf8]{inputenc}
\usepackage[T1]{fontenc}
\usepackage[polish]{babel}
\usepackage{graphicx}
\usepackage{geometry}
\usepackage{hyperref}
\usepackage{float}
\usepackage{titlesec}
\usepackage{xcolor}

% Konfiguracja marginesów
\geometry{
 a4paper,
 total={170mm,257mm},
 left=20mm,
 top=20mm,
}

% Konfiguracja linków
\hypersetup{
    colorlinks=true,
    linkcolor=black,
    filecolor=magenta,      
    urlcolor=blue,
}

\title{\textbf{Instrukcja konfiguracji połączenia zdalnego (VPN + WinSCP)}}
\author{Maria Junak} 
\date{\today}

\begin{document}

\maketitle

\begin{abstract}
Niniejszy dokument opisuje procedurę łączenia się z siecią Instytutu Matematycznego. Zawiera konfigurację VPN, diagnostykę połączenia (ping) oraz próbę nawiązania sesji SFTP, wraz z analizą napotkanych problemów z uwierzytelnianiem.
\end{abstract}

\tableofcontents
\newpage

\section{Konfiguracja i uruchomienie VPN}

Aby uzyskać dostęp do wewnętrznej sieci laboratoriów (adresacja 192.168.x.x), konieczne jest zestawienie tunelu VPN.

\subsection{Dodawanie połączenia w systemie Windows}
W ustawieniach sieci VPN dodano nową konfigurację dla serwera Instytutu.
\begin{itemize}
    \item \textbf{Adres serwera:} \texttt{156.17.86.7}
    \item \textbf{Typ:} Automatyczny / L2TP
\end{itemize}

\begin{figure}[H]
    \centering
    \includegraphics[width=0.7\textwidth]{screenshots/vpn_setup.png}
    \caption{Konfiguracja parametrów połączenia VPN.}
\end{figure}

Po zatwierdzeniu ustawień, nawiązano połączenie. Status "Połączono" świadczy o poprawnym zestawieniu tunelu.

\begin{figure}[H]
    \centering
    \includegraphics[width=0.5\textwidth]{screenshots/vpn_connected.png}
    \caption{Prawidłowo zestawione połączenie VPN.}
\end{figure}

\section{Diagnostyka sieci (Ping)}

Przed próbą transferu plików przeprowadzono diagnostykę sieci, aby zweryfikować widoczność serwerów docelowych.

\begin{figure}[H]
    \centering
    \includegraphics[width=0.8\textwidth]{screenshots/failed_ping.png}
    \caption{Diagnostyka połączenia poleceniem PING. Ilustracja braku dostępu lub problemów z routingiem przed poprawnym zestawieniem tunelu VPN.}
    \label{fig:ping}
\end{figure}

\newpage

\section{Połączenie SFTP (WinSCP)}

Kolejnym krokiem była konfiguracja klienta WinSCP w celu przesyłania plików.

\subsection{Parametry sesji}
Skonfigurowano połączenie z serwerem wewnątrz sieci laboratoryjnej:
\begin{itemize}
    \item \textbf{Host:} \texttt{192.168.50.144}
    \item \textbf{Protokół:} SFTP
\end{itemize}

\begin{figure}[H]
    \centering
    \includegraphics[width=0.7\textwidth]{screenshots/winscp_login.png}
    \caption{Okno logowania programu WinSCP.}
\end{figure}

\subsection{Weryfikacja uwierzytelniania}
Podjęto próbę zalogowania się do serwera. Jak widać na poniższym zrzucie ekranu, **połączenie sieciowe zostało nawiązane poprawnie** (serwer odpowiedział), jednak wystąpił błąd autoryzacji.

\begin{figure}[H]
    \centering
    \includegraphics[width=0.85\textwidth]{screenshots/winscp_main.png}
    \caption{Próba logowania zakończona błędem "Access denied". Komunikat potwierdza, że VPN działa (widzimy serwer), a problem leży jedynie po stronie błędnego hasła lub loginu.}
\end{figure}

\section{Wnioski}
Testy potwierdziły, że konfiguracja VPN jest poprawna i umożliwia komunikację z siecią wewnętrzną (co potwierdza odpowiedź serwera przy próbie logowania). Do pełnego uzyskania dostępu do plików wymagane jest jedynie zresetowanie hasła lub uzyskanie poprawnych danych logowania do konta w pracowni.

\end{document}